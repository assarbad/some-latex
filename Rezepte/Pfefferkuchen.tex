% !TeX encoding = UTF-8
% !TeX spellcheck = de_DE_OLDSPELL
% !TEX TS-program = lualatex
\ifdefined\directlua\else
    \errmessage{LuaTeX is required to typeset this document}
    \csname @@end\expandafter\endcsname
\fi
\documentclass[paper=A4,fontsize=12pt,DIV=18,fraktur=false]{rezepte}
\usepackage[autostyle]{csquotes} %% [autostyle,german=guillemets]
\setmainlanguage{german}

\begin{document}

\section*{Omas Pfefferkuchen}

\subsection*{Zutaten}

\begin{itemize}
    \item 3~℔ Weizenmehl, Type 405 oder doppelgriffig (\enquote{Inſtant-Mehl})
    \item 400~g Zucker
    \item 60~g Pfefferkuchengewürz\footnote{Oſtmann: 4 Tüten à 15~g, Kotányi: 2 Tüten à 30~g etc.}
    \item 3 Eier und 3 Eigelbe\\
        Eiweiß zum Beſtreichen vor dem Backen aufheben
    \item 30-35~g Pottaſche in ein wenig Milch aufgelöſt
    \item 300~g Kokosfett, handwarm geſchmolzen
    \item 450~g Zuckerrübenſirup, handwarm
    \item Wahlweiſe: 2~EL Rum
\end{itemize}

\subsection*{Zubereitung}

\begin{itemize}
    \item Teig aus den Zutaten in angegebener Reihenfolge kneten
    \begin{itemize}
        \item Mehl, Zucker und Pfefferkuchengewürz gut vermiſchen\footnote{Man kann alternativ ein paar Tage vorher das Kokosfett ſchmelzen und die Gewürze hineingeben. Dann ein paar Tage kühl ſtellen bevor der Teig geknetet wird.}
        \item Eier und Eigelbe mit der Mischung \enquote{verreiben}
        \item Pottaſche in Milch auflöſen und ebenfalls \enquote{verreiben}
        \item Flüſſiges Kokosfett hinzugeben und möglichſt gut miſchen
        \item Abſchließend mit Zuckerrübenſirup alles zu einem derben Teig kneten
    \end{itemize}
    \item Teig kurz kühlſtellen, dann läßt er ſich beſſer ausſtechen
    \item Teig etwa 1 bis 1,5~cm gleichmäßig dick ausrollen und Pfefferkuchen ausſtechen\footnote{Nach dem Ausſtechen können ſie auch gut bis zu drei Tage kühl ſtehen; dies funktioniert aber wohl nur bei Verwendung von Pottaſche, nicht falls ein anderes Backtriebmittel als Erſatz benutzt wurde. Hirſchhornſalz in gleicher Menge funktioniert, jedoch laufen die Pfefferkuchen beim Backen mehr auseinander.}
    \item Vor dem Backen die Pfefferkuchen mit Eiweiß beſtreichen, um den fertigen Pfefferkuchen eine glänzende Oberfläche zu geben
    \item 11-13~Minuten bei 175°C (Umluft) backen
\end{itemize}

\subsection*{Lagerung}

Pfefferkuchen luftdicht verpackt bei Zimmertemperatur lagern.

\end{document}
