% !TeX encoding = UTF-8
% !TeX spellcheck = de_DE_OLDSPELL
% !TEX TS-program = lualatex
\ifdefined\directlua\else
    \errmessage{LuaTeX is required to typeset this document}
    \csname @@end\expandafter\endcsname
\fi
\documentclass[paper=A4,fontsize=12pt,DIV=18,fraktur=false]{rezepte}
\usepackage[autostyle]{csquotes} %% [autostyle,german=guillemets]
\setmainlanguage{german}

\begin{document}

\section*{Omas Mürbeplätzchen}

\subsection*{Zutaten}

\begin{itemize}
    \item 2~℔ Weizenmehl\\
    Type 405 oder doppelgriffig (\enquote{Instant-Mehl})\footnote{Andere Mehlſorten verfälſchen das Ergebnis.}
    \item 400~g Kriſtallzucker\footnote{Originalrezept verlangt 500~g, bis 350~g wurde ſchon probiert. Achtung: weniger Zucker bedeutet hier geringere Haltbarkeit.}, je feinkörniger deſto beſſer
    \item Buttervanille entſprechend der Mehlmenge
    \item 1-2 geſtrichene Teelöffel Salz
    \item 1~℔ Butter\footnote{Kein Butterſchmalz/-reinfett, wir brauchen die bindenden/emulgierenden Anteile der Butter!}, ſehr weich oder geſchmolzen, aber höchſtens handwarm
    \item 13-14 Eigelbe\footnote{Etwas Eiweiß aufheben zum ſpäteren Beſtreichen der Plätzchen.} (ca. 210..250~g)\footnote{Je mehr Eigelbe, deſto mürber. Je mürber deſto kühler ſollte der Teig verarbeitet werden.}
\end{itemize}

\subsection*{Zubereitung}

\begin{itemize}
    \item Eigelbe, Zucker, Salz und Buttervanille mit einem Rührgerät gut vermiſchen. Dann Butter hinzugeben und nochmals gründlich vermiſchen.
    \item Mehl zu der Miſchung geben und gut durchkneten.
    \item Optional den Teig für mindeſtens eine halbe Stunde kaltſtellen. Das hilft bei beim Ausſtechen, da der Teig damit feſter wird. Kaltſtellen erſchwert jedoch auch das Ausrollen.
    \item Einen Teil des Eiweißes leicht ſchlagen und vor dem Backen auf die Plätzchen ſtreichen\footnote{Es ſoll nicht ſchaumig werden, ſondern nur weniger dickflüſſig.}. Dies ſorgt für eine glänzende Oberfläche.
    \item Bei 200°C Umluft etwa 10~Minuten backen, bis ſie anfangen gold-braun zu werden.
\end{itemize}

\subsection*{Lagerung}

Die Plätzchen möglichſt luftdicht und vor Feuchtigkeit geſchützt bei Zimmertemperatur oder leicht kühl aufbewahren. Halten ſich erfahrungsgemäß mehrere Wochen.

\end{document}
