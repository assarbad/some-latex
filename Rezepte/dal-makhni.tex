\documentclass[version=last,fontsize=11pt,paper=A4]{scrartcl}
%%\usepackage[german]{babel}
\usepackage{csquotes}
\usepackage{microtype}
\usepackage{luaotfload}
\directlua{luaotfload.add_fallback("genericfallback", {"NotoSerif:mode=node;"})}
\usepackage{fontspec}
\usepackage{polyglossia}
\setmainlanguage{german}
\setmainfont{TeX Gyre Bonum}[RawFeature={fallback=genericfallback}]
\setsansfont{TeX Gyre Heros}
\setmonofont{TeX Gyre Cursor}

\begin{document}
\thispagestyle{empty}

% Quelle: https://recipes.instantpot.com/recipe/three-lentil-dal-makhani/

\section*{Dal Makhni aus dem Instant Pot}

\subsection*{Zutaten}

\begin{itemize}
    \item 2 Eßlöffel Butter oder Butterschmalz
    \item 2 Eßlöffel Knoblauch, gehackt
    \item 2 gehäufte Löffel frisch geriebener Ingwer
    \item 1 Teelöffelspitze Kreuzkümmel
    \item 1 Eßlöffel Garam Masala\\
          alternativ 1 gehäufter Eßlöffel Garam-Masala-Paste
    \item 1 Teelöffel Gelbwurzpulver\footnote{englisch Turmeric}
    \item 1 Teelöffel (gestrichener) Salz
    \item 1 Teelöffelspitze Cayenne-Pfeffer
    \item 1 Zimtstange
    \item 4 Kardamomkapseln (grün oder weiß)
    \item 2 Lorbeerblätter
    \item 1 Dose gehackte Tomaten\\
          alternativ 400 bis 450 g frische gehackte Tomaten
    \item 300 g Auswahl an Linsen, bspw.
    \begin{itemize}
        \item 200 g geschälte Uridlinsen + 100 g Berglinsen\footnote{Statt Berglinsen gehen im Zweifel auch grüne Linsen, bzw. Tellerlinsen.}
        \item 100 g Uridlinsen + 100 g Belugalinsen + 100 g Berglinsen
        \item 200 g rote Linsen + 100 g Berglinsen
        \item 300 g geteilte Uridlinsen\footnote{Mit Schale, aber diese ist geöffnet. Entspricht am ehesten dem Original.}
    \end{itemize}
    \item 1 Liter Wasser
    \item 150 bis 250 g Crème fraîche, je nach Verfügbarkeit\\
          Schlagsahne geht auch, aber dann wird es dünnflüssiger.
\end{itemize}

\subsection*{Zubereitung}

\begin{enumerate}
    \item Sauté am Instant Pot setzen und auf 5 Minuten einstellen
    \item Butter/Butterschmalz schmelzen
    \item Gewürze hinzugeben und \enquote{anrösten} bis sie duften
    \item Tomaten hinzugeben und rühren bis sie zu zerfallen beginnen, 1 bis 2 Minuten
    \item Sauté ausschalten und Linsen einrühren
    \item Wasser hinzugeben und Topf schließen
    \item Modus \enquote{Meat/Stew} oder \enquote{Pressure cook} einstellen und auf hohem Druck (\enquote{High pressure}) für 20 Minuten einstellen
    \item \enquote{Quick-release}-Methode benutzen um den Druck zu mindern und Topf danach öffnen.
    \item Zimtstange, Kardamomkapseln und Lorbeerblätter suchen und wegwerfen
    \item Crème fraîche (oder Sahne) gut einrühren und nochmals kurz ziehen lassen
\end{enumerate}

\end{document}
