\documentclass[version=last,paper=A4]{scrartcl}
%%\usepackage[german]{babel}
\usepackage{microtype}
\usepackage{luaotfload}
\directlua{luaotfload.add_fallback("genericfallback", {"NotoSerif:mode=node;"})}
\usepackage{fontspec}
\usepackage{polyglossia}
\usepackage{csquotes}
\setmainlanguage{german}
\setmainfont{TeX Gyre Bonum}[RawFeature={fallback=genericfallback}]
\setsansfont{TeX Gyre Heros}
\setmonofont{TeX Gyre Cursor}

\begin{document}
\thispagestyle{empty}

\section*{Mürbeplätzchen}

\subsection*{Zutaten}

\begin{itemize}
    \item 2~℔ Weizenmehl\\
    Type 405 oder doppelgriffig (\enquote{Instant-Mehl})\footnote{Andere Mehlsorten verfälschen das Ergebnis.}
    \item 350~g Kristallzucker, je feinkörniger desto besser
    \item Buttervanille entsprechend der Mehlmenge
    \item 2 gestrichene Teelöffel Salz
    \item 1~℔ Butter\footnote{Kein Butterschmalz, wir brauchen die bindenden Anteile der Butter, kein Reinfett!}, sehr weich oder geschmolzen, aber höchstens handwarm
    \item 13-14 Eigelbe (ca. 230..250~g)
\end{itemize}

\subsection*{Zubereitung}

\begin{itemize}
    \item Mehl, Zucker, Buttervanille und Salz gründlich vermengen. Danach die Butter untermengen und \enquote{verreiben} um ein möglichst gute Verteilung zu gewährleisten. Zuletzt mit den Eigelben zu einem gleichmäßigen Teig verkneten.
    \item Optional den Teig für mindestens eine halbe Stunde kaltstellen. Das hilft bei beim Ausstechen, da der Teig damit fester wird. Kaltstellen erschwert jedoch auch das Ausrollen.
    \item Einen Teil des Eiweißes leicht schlagen und vor dem Backen auf die Plätzchen streichen\footnote{Es soll nicht schaumig werden, sondern nur weniger dickflüssig.}. Dies sorgt für eine glänzende Oberfläche.
    \item Bei 200°C Umluft etwa 10~Minuten backen, bis sie anfangen gold-braun zu werden.
\end{itemize}

\subsection*{Lagerung}

Die Plätzchen möglichst luftdicht und vor Feuchtigkeit geschützt bei Zimmertemperatur oder leicht kühl aufbewahren. Halten sich erfahrungsgemäß mehrere Wochen.

\end{document}
