% !TeX encoding = UTF-8
% !TeX spellcheck = en_US
\documentclass[fontsize=11pt,paper=A4,landscape,DIV=24,BCOR=0pt,parskip=false,version=last]{scrartcl}
%%\usepackage{lua-visual-debug}
\usepackage{fontspec-luatex}
\usepackage{libertine}
\usepackage{multicol}
\usepackage{csquotes}
\usepackage[normalem]{ulem}
\usepackage{tabularray}
\usepackage[hidelinks]{hyperref}
\usepackage{microtype}
\usepackage{pdfpages}

\setmainfont{Linux Libertine O}
\setsansfont{Linux Biolinum O}
\setmonofont{Linux Libertine Mono O}

\setlength{\parindent}{0pt}
%\RedeclareSectionCommand[beforeskip=-5.5ex plus -1ex minus -.2ex,afterskip=1.3ex plus -.2ex]{section}
%\setparsizes{0pt}{0pt}{0pt plus 1fil}
%\setlength{\parfillskip}{0pt plus 1fil}

%% For the usual 2-column table
\NewTblrEnviron{kbdtblr}
\SetTblrInner[kbdtblr]{
    colspec={X[1,m]X[2,m,font=\small]},
    row{odd} = {bg=gray9},
    row{1}   = {bg=azure8,font=\large\sffamily\bfseries},
}
\newcommand{\mnem}[1]{\underline{\textbf{#1}}}
\newcommand{\LKeyShiftF}[1]{\LKeyShift+\LKeyF{#1}}
\newcommand{\LKeyCtrlF}[1]{\LKeyCtrl+\LKeyF{#1}}
\newcommand{\hexrays}{\textsf{\tiny[Hex-Rays]}}

\begin{document}
\pagestyle{empty} % suppress the page number etc.

\begin{multicols}{4}

\begin{kbdtblr}{rl}
\SetCell[c=2]{c} File Operations \\
\LKeyCtrlF{9} & Parse C header file \\
\LKeyAltF{10} & Create ASM file \\
\LKeyCtrlX{W} & Save database \\
{\LKeyAltF{4} \LKeyAltX{X}} & Exit and save \\
\end{kbdtblr}

\begin{kbdtblr}{rl}
\SetCell[c=2]{c} Navigation \\
\LKeyEnter & Jump to operand \\
\LKeyAlt+\LKeyEnter & Jump in a new window \\
\LKeyEsc & Jump back \\
\LKeyCtrl+\LKeyEnter & Jump forward \\
\LKey{G} & \mnem{G}o to address \\
\LKeyCtrlX{L} & Jump by \mnem{l}abel (name) \\
\LKeyCtrlX{P} & Jump by function (\mnem{p}roc) \\
\LKeyCtrlX{S} & Jump to \mnem{s}egment \\
\LKeyCtrlX{G} & Jump to segment register \\
\LKeyCtrlX{X} & List \mnem{x}refs \emph{to} ... \\
\LKey{X} & Jump to \mnem{x}ref to operand \\
\LKeyTab & Jump to pseudocode \\
\LKeyCtrlX{E} & Jump to \mnem{e}ntry point \\
\LKeyAltX{M} & \mnem{B}ookmark position \\
\LKeyCtrlX{M} & Jump to \mnem{b}ookark \\
\LKeyCtrlX{Q} & Jump to problem \\
\LKeyCtrlX{F} & Erroneous operand \\
\end{kbdtblr}

\begin{kbdtblr}{rl}
\SetCell[c=2]{c} Search \\
\LKeyAltX{C} & Next \mnem{c}ode \\
\LKeyCtrlX{D} & Next \mnem{d}ata \\
\LKeyCtrlX{A} & Next explored \\
\LKeyCtrlX{U} & Next \mnem{u}nexplored \\
\LKeyAltX{I} & \mnem{I}mmediate value \\
\LKeyCtrlX{I} & Next \mnem{i}mmediate value \\
\LKeyAltX{T} & \mnem{T}ext \\
\LKeyCtrlX{T} & Next \mnem{t}ext \\
\LKeyAltX{B} & Sequence of \mnem{b}ytes \\
\LKeyCtrlX{B} & Next sequence of \mnem{b}ytes \\
\end{kbdtblr}

\begin{kbdtblr}{rl}
\SetCell[c=2]{c} Edit \\
\LKey{N} & Rename \\
\LKey{C} & Make code \\
\LKey{D} & Make string \\
\LKey{D} & Make data / cycle types \\
\LKey{U} & Undefine \\
\LKey{asterisk} or \LKeyPad{13} & Make array \\
\LKey{O} & Offset (data segment) \\
\LKeyCtrlX{O} & Offset (current segment) \\
\LKeyAltX{R} & Offset by (any segment) \\
\LKeyCtrlX{R} & Offset (user-defined) \\
\LKey{numbersign} & Number (default) \\
\LKey{Q} & Hexadecimal \\
\LKey{H} & Decimal \\
\LKey{B} & \mnem{B}inary \\
\LKey{R} & Cha\mnem{r}acter \\
\LKey{S} & \mnem{S}egment \\
\LKey{M} & Enum \mnem{m}ember \\
\LKey{K} & Stac\mnem{k} variable \\
\LKey{underscore} & Change sign \\
\LKey{asciitilde} & Bitwise negate \\
\LKeyAltX{A} & String literals \\
\LKeyAltX{D} & Setup data types \\
\LKeyAltX{S} & Edit segment \\
\LKeyAltX{G} & Change segment register \\
\LKeyAltX{Q} & Struct variable \\
\LKeyAltX{Y} & Select union member \\
\LKeyCtrlX{Z} & Undo \\
\LKeyShiftCtrlX{Z} & Redo \\
\end{kbdtblr}

\begin{kbdtblr}{rl}
\SetCell[c=2]{c} Comments \\
\LKey{semicolon} & Repeatable comment \\
%% On first instruction becomes the function comment
\LKey{colon} & Comment \\
\LKeyIns & Anterior lines \\
%% Also works on database comment at the top
\LKeyShift+\LKeyIns & Posterior lines \\
\LKey{slash} & {\hexrays} Line comment
\end{kbdtblr}

\begin{kbdtblr}{rl}
\SetCell[c=2]{c} Subviews \\
\LKeyShiftF{1} & Open \emph{Local types} \\
\LKeyShiftF{3} & Open \emph{Functions} \\
\LKeyShiftF{4} & Open \emph{Names} \\
\LKeyShiftF{5} & Open \emph{Signatures} \\
\LKeyShiftF{7} & Open \emph{Segments} \\
\LKeyShiftF{8} & Open \emph{Segment registers} \\
\LKeyShiftF{9} & Open \emph{Structures} \\
\LKeyShiftF{10} & Open \emph{Enumerations} \\
\LKeyShiftF{11} & Open \emph{Type libraries} \\
\LKeyShiftF{12} & Open \emph{Strings} \\
\LKeyAltF{9} & Recent scripts \\
\end{kbdtblr}

\begin{kbdtblr}{rl}
\SetCell[c=2]{c} Functions (aka procedures) \\
\LKey{X} & Create function (\mnem{p}roc) \\
\LKeyAltX{P} & Edit function (\mnem{p}roc) \\
\LKey{E} & Set \mnem{e}nd of function \\
\LKeyCtrlX{K} & Stac\mnem{k} variables \\
\LKeyAltX{K} & Change stac\mnem{k} pointer \\
\LKey{V} & Rename register \\
\LKey{Y} & Set function t\mnem{y}pe \\
\end{kbdtblr}

\begin{kbdtblr}{rl}
\SetCell[c=2]{c} Lumina (metadata service) \\
\LKeyF{12} & Pull all metadata \\
\LKeyCtrlF{12} & Push all metadata \\
\LKeyAltF{12} & View all metadata \\
\end{kbdtblr}

\begin{kbdtblr}{rl}
\SetCell[c=2]{c} Debugger \\
\LKeyF{2} & Add breakpoint \\
\LKeyF{9} & Start process \\
\LKeyCtrlF{2} & Terminate process \\
\LKeyF{7} & Step into \\
\LKeyF{8} & Step over \\
\LKeyCtrlF{7} & Run until return \\
\LKeyF{4} & Run to cursor \\
\LKeyCtrlAltX{B} & Breakpoint list \\
\LKeyCtrlAltX{S} & Stack trace \\
\end{kbdtblr}

\begin{kbdtblr}{rl}
\SetCell[c=2]{c} Dialog Boxes \\
{\LKeyTab \\ \LKeyShift{}+\LKeyTab} & Navigate \\
\LKeySpace && Toggle \\
{\LKeyEnter \\ \LKeyAltX{K} \\ \LKeyCtrl{}+\LKeyEnter} & Confirm \\
{\LKeyEsc \\ \LKeyAltF{4}} & Cancel \\
\end{kbdtblr}

\begin{kbdtblr}{rl}
\SetCell[c=2]{c} Miscellaneous \\
\LKey{question} & Evaluate expression (calculator) \\
\LKeyCtrl{}+\LKeyTab & Window list (next) \\
\LKeyAltX{1} ... \LKeyAltX{9} & Switch to window \#1...9 \\
\end{kbdtblr}

\end{multicols}

%% \url{https://hex-rays.com/blog/igor-tip-of-the-week-01-lesser-known-keyboard-shortcuts-in-ida/}

\end{document}